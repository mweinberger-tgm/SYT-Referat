\documentclass[letterpaper, 12pt]{article}

%%%%%%%%%%%%%%%%%%%%%%%%%%%%%
% DEFINITIONS
% Change those informations
% If you need umlauts you have to escape them, e.g. for an ü you have to write \"u
\gdef\mytitle{Ausarbeitung}
\gdef\mythema{Synchronisierung \& Konsistenz}

\gdef\mysubject{SYT}
\gdef\mycourse{5BHIT 2015/16}
\gdef\myauthor{Erik Br\"andli \& Michael Weinberger}

\gdef\myversion{0.1}
\gdef\mybegin{6. November 2015}
\gdef\myfinish{??.??.2016}

\gdef\mygrade{}
\gdef\myteacher{}
%
%%%%%%%%%%%%%%%%%%%%%%%%%%%%%

\input special/preamble.tex

\let\tempsection\section
\renewcommand\section[1]{\vspace{-0.3cm}\tempsection{#1}\vspace{-0.3cm}}
\WithSuffix\newcommand\section*[1]{\tempsection*{#1}}

\let\tempsubsection\subsection
\renewcommand\subsection[1]{\vspace{0cm}\tempsubsection{#1}\vspace{0cm}}

\let\tempsubsubsection\subsubsection
\renewcommand\subsubsection[1]{\vspace{0cm}\tempsubsubsection{#1}\vspace{0cm}}

\linespread{0.94}

\lhead{\mysubject}
\chead{}
\rhead{\bfseries\mythema}
\lfoot{\mycourse}
\cfoot{\thepage}
% Creative Commons license BY
% http://creativecommons.org/licenses/?lang=de
\rfoot{\ccby\hspace{2mm}\myauthor}
\renewcommand{\headrulewidth}{0.4pt}
\renewcommand{\footrulewidth}{0.4pt}

\begin{document}
\parindent 0pt
\parskip 6pt

\pagenumbering{Roman} 
\input{special/title}

\clearpage
\thispagestyle{empty}
\tableofcontents

\newpage
\pagenumbering{arabic}
\pagestyle{fancy}

%\vspace{-0.5cm}
\section{Disaster Recovery}

\subsection{Einführung, grober Überblick \cite{bookone} \cite{intro} \cite{intro2}}
Disaster Recovery (dt. auch Katastrophenwiederherstellung), im Folgenden auch \textit{DR} genannt, beschreibt die Vorbereitung und Reaktion auf sogenannte Katastrophen, die abgespeicherte Daten und Lauffähigkeit eines IT-Systems betreffen. In diesem Bereich der Sicherheitsplanung ist mit negativen Ereignissen all das gemeint, was den Betrieb eines Unternehmens gefährdet. Hierzu gehören Cyberattacken, Infrastrukturausfälle ebenso wie Naturkatastrophen. DR umfasst beispielsweise Schritte zur Wiederherstellung von Server oder Mainframes mit Backups oder ferner die Bereitstellung von LANs für die unmittelbaren geschäftlichen Bedürfnisse. 

\subsubsection{Disaster Recovery Plan}

In dessen Folge dokumentiert ein Disaster Recovery Plan, im Folgenden auch DRP genannt, dann konkret Richtlinien, Verfahren und Maßnahmen, um die Störung eines Unternehmens im Falle eines Desasters zu begrenzen. Wie bei einer Katastrophe macht das Ereignis die Fortführung des normalen Geschäftsbetriebs unmöglich. \\
Falls ein DRP besteht, kann das Unternehmen die Auswirkungen des Desasters minimieren und ihre geschäftskritischen Prozesse schnell fortführen. Die Disaster-Recovery-Planung beinhaltet in der Regel eine Analyse der Geschäftsprozesse und des Bedarfs. Sie kann auch einen Schwerpunkt zur Prävention beinhalten. Disaster Recovery ist ein wichtiger Aspekt von Enterprise-Computing. Die Unterbrechung des Dienstes oder der Verlust von Daten kann sich schwerwiegend auf die Finanzen auswirken, sei es direkt oder durch den Imageverlust.

\subsubsection{Business Continuity}

Business Continuity beschreibt Prozesse und Verfahren eines Unternehmens, die die Weiterführung von wichtigen Geschäftsprozessen während und nach einem Desaster sichern sollen. Dabei liegt der Schwerpunkt mehr auf der Aufrechterhaltung der Geschäftstätigkeit als bei der Infrastruktur. Business Continuity und Disaster Recovery sind eng verbunden, so dass beide Begriffe manchmal kombiniert werden.

\subsubsection{Arten von Katastrophen}

Wie bereits kurz erwähnt, 

\clearpage

\bibliographystyle{unsrt}
\bibliography{Synchronisierung_Konsistenz_Braendli_Weinb_5BHIT}
\lstlistoflistings
\listoffigures

\end{document}
